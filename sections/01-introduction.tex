\chapter{Introdução}
\label{chap:intro}

\section{Objetivo}

O projeto tem como objetivo construir um robô bípede de pequeno porte capaz de se movimentar utilizando os pés em um determinado espaço no laboratório. Além disso, o robô deve ser capaz de reconhecer marcos fiduciais para realizar missões que são enviadas ao seu processamento,bem como desviar de obstáculos que impedem a sua movimentação.

\section{Justificativa}

Robôs antropomórficos são amplamente utilizados em diversas áreas do dia-a-dia, desde interações com humanos até aplicações na área da saúde, bem como em pesquisas acadêmicas, sendo uma das configurações mais eficiente para locomoção de ambientes de difícil navegação.
A vantagem da locomoção por pernas é a utilização de passos discretos para o equilíbrio e a movimentação do robô, o que permite que este realize manobras em terrenos acidentados e escadas. 

\section{Motivação}

A motivação do projeto consiste em tornar-se apto a trabalhar com robótica através da construção de um robô bípede, e despertar interesse de pesquisa na área de robótica, bem como desenvolver habilidades essenciais de engenharia para surpir as grandes demandas da industria 4.0.
